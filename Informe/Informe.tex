\documentclass[10pt]{article}
% Language setting
% Replace `english' with e.g. `spanish' to change the document language
\usepackage[spanish]{babel}

% Set page size and margins
% Replace `letterpaper' with`a4paper' for UK/EU standard size
\usepackage[a4paper,top=1.9cm,bottom=3.67cm,left=1.9cm,right=1.32cm,marginparwidth=1.75cm]{geometry}

% Useful packages
\usepackage{amsmath}
\usepackage{graphicx}
\usepackage[colorlinks=true, allcolors=blue]{hyperref}

% Set document Font
\usepackage{fontspec}
\setmainfont{Times New Roman}

\title{Proyecto Global Integrador AyME:\\
Control de Accionamiento de CA con Motor Sincrónico de Imanes Permanentes}

\author{Guarise Renzo, Trubiano Lucas\\
Profesor: Ing. Gabriel L. Julián\\
\\
Universidad Nacional de Cuyo - Facultad de Ingeniería\\
Automática y Máquinas Eléctricas\\
Ingeniería Mecatrónica}

\begin{document}
\maketitle

\begin{center} % Centrar texto
    {\Large \textbf{Resumen}}
\end{center}

Resumen sobre el proyecto.

Al final del resumen empezamos con el resto del informe

\newpage

% \tableofcontents % Creamos el indice

\section{Introducción}

\section{Desarrollo}
\subsection{Modelado, Análisis y Simulación dinámica del SISTEMA FÍSICO a “Lazo Abierto” (Sin Controlador externo de Movimiento)}
\subsubsection{Modelo matemático equivalente (1 GDL) del subsistema mecánico completo}



\subsubsection{Modelo dinámico del sistema físico completo}

\begin{enumerate}
    \renewcommand{\theenumi}{\alph{enumi}} %Letras minúsculas
    \item \textbf{Modelo global no lineal (NL)}
    
    TEXTO

    \item \textbf{Linealización Jacobiana}
    \vspace{0.3cm} 
    \\Primero obtendremos el modelo \textbf{global linealizado con parámetros variables (LPV)}, para $ i_{ds}^{r}\left( t \right ) \neq 0 $ (caso gral.). 
    Un sistema \textbf{no lineal} puede ser representado por:
    \begin{equation}
        \dot{x}\left ( t \right )= f\left ( x\left ( t \right ),u\left ( t \right ) \right ) \ \ \ ,\ \ \ x\left ( t_{0} \right )=x_{0}
    \end{equation}
    \begin{equation}
        y\left ( t \right )=h\left ( x\left ( t \right ) \right )
    \end{equation}
    donde f es una función vectorial de n × 1 elementos que esta en función de un vector x conformado por las variables de estado.
    n representa el orden del sistema y la solución de x(t) representa una curva en el espacio de estado, denominada como trayectoria de estado.
    Asumiendo que toda variable estará definida como,
    \begin{equation}
        z\left ( t \right )\equiv Z_{0} + \Delta z\left ( t \right )
    \end{equation}
    donde Zo es una magnitud cuasi-estacionaria de variación muy lenta con el tiempo y ∆z(t) una magnitud pequeña de variación rápida en el tiempo.
    El sistema queda expresado:
    \begin{equation}
        \begin{cases}
            \dot{x}\left ( t \right )\equiv
            \frac{\mathrm{d} X_{o}\left ( t \right )}{\mathrm{d} t}+
            \frac{\mathrm{d} \Delta x\left ( t \right )}{\mathrm{d} t}
            =
            f\left ( X_{o}\left ( t \right )+\Delta x\left ( t \right ),U_{o}\left ( t \right )+\Delta u\left ( t \right ) \right )
            \\
            X_{o}\left ( 0 \right )=x_{0}\ \ ;\ \ \ \Delta x\left ( 0 \right )=0
        \end{cases}
    \end{equation}

    \item Linealización por Realimentación NL
    \begin{itemize}
        \item Determinación de la Restricción o Ley de Control mínima
        \item Restricción o Ley de Control complementaria mínima en el eje q
    \end{itemize}
    \item Comparación del modelo dinámico LTI equivalente aumentado vs. el modelo dinámico global LPV
\end{enumerate}


\subsubsection{Análisis de Estabilidad a lazo abierto para el modelo LTI equivalente aumentado}
\subsubsection{Análisis de Observabilidad completa de estado para el modelo LTI equivalente aumentado}
\subsubsection{Análisis de Controlabilidad completa de estado para el modelo LTI equivalente aumentado}
\subsubsection{Simulación dinámica en DT, comparando el modelo NL completo desacoplado con Ley de control
NL vs LTI equivalente aumentado}


\subsection{Diseño, Análisis y Simulación con CONTROLADOR de Movimiento en Cascada con Modulador de Torque equivalente (Control Vectorial)}
\subsubsection{Modulador de Torque equivalente (Controlador interno vectorial de corriente/torque)}
\subsubsection{Controlador externo de movimientos: posición/velocidad }
\subsubsection{IncorporaciónydiseñodeObservadordeEstadodeordenreducidosóloparalapartemecánica de este controlador}
\subsubsection{Simulación en tiempo continuo con modelo completo NL}
\subsubsection{Verificacióndedesempeñoy/omejoras}

\section{Conclusiones}

\section{Referencias}

\end{document}